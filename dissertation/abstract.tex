\begin{abstract}
A reactive program is characterized by ``ongoing interactions with its environment,'' asynchrony, and concurrency.
Reactive systems are already a critical part of our infrastructure in the form of embedded systems, communication networks, and enterprise systems.
Reactive systems, specifically embedded systems, networked systems, and interactive systems, are poised to continue proliferating due to advances in hardware and software platforms.
A predicted increase in the complexity of these systems combined with the accidental complexity already associated with existing approaches necessitate a new model and platform for reactive systems.

We propose three contributions to the state of the art in reactive system development.
First, we propose a model for reactive systems that 1)~reduces the accidental complexity associated with their design and implementation by encoding their semantics directly and 2) ensures that decomposition and composition can be applied to their design and implementation in a principled way.
The model is based on the direct manipulation of state, implicit atomicity, and non-deterministic sequencing.
Second, we propose a programming language based on the model.
The main challenges of implementing the model lie in the checks that enforce the concurrency semantics of the model.
Third, we propose to evaluate the model and platform by applying the model to the design and development of an embedded web server and evaluating a number of micro-benchmarks to quantify the scheduling overhead of the run-time environment.
The goal of the evaluation is to gauge the model as a practical software engineering tool by gaining experience about the strengths and weaknesses of the model and the associated costs of using the model.
\end{abstract}
