\chapter{Inter-Component Communication}

%% \section{Introduction}




%% Inter-component communication (ICC) refers to the mechanisms that allow control and data to be transferred between reactive components.
%% The basic semantics of port calls and trigger statements are described in~\ref{model}.
%% The goal of this chapter is to expand upon these semantics in the context of a programming language implementation.

%% \emph{port call} and \emph{trigger statement}

%% A port call links the execution of an action or reaction in one component with a reaction in another component.
%% The former is referred to as the \emph{caller} while the latter is the \emph{callee}.
%% The port call has two primary functions:
%% \begin{itemize}
%% \item \emph{Communication} - A port call allows the caller to communicate with the callee by copying, sharing, or transferring data.
%% \item \emph{Transfer of Control} - A port call kicks off the immutable phase of the callee.
%% \end{itemize}

%% A trigger statement synchronizes




%%  while a trigger statement synchronizes the execution of a mutablecomponents


%% A port call is bound to a trigger statement and occurs in the immutable phase of an execution.


%% The port call bears a similarity


%% by which control and data are transferred between reactive components.
%% The mechanism in question is the \emph{port call}


%% ICC is comparable to inter-process communication (IPC) refers to the mechanisms by which control and data are transferred between processes in a traditional process-oriented operating system.


%% Whereas IPC includes a variety of mechanisms, such as signals, shared memory segments, message queues, and sockets, ICC includes only one:  the \emph{port call}.
%% The basic semantics of the port call are discussed in \ref{model}.

%% \section{Pass-By-Value}

%% \section{Pass-By-Reference}

%% \section{Transferrable Objects}
