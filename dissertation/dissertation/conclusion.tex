\chapter{Conclusion}

A reactive system is characterized by ``ongoing interactions with its environment\cite{manna1992temporal}.''
Asynchrony and concurrency are two features of reactive systems that make them inherently difficult to develop.
Reactive systems are already used in critical infrastructure and the number, diversity, and scale of reactive systems is expected to increase given the continuing proliferation of embedded, networked, and interactive systems.
Decomposition and composition are two complementary techniques that we would like to use when designing and implementing reactive systems given such increases in complexity.
We argue that the techniques based on explicit atomicity and deterministic sequencing prevent decomposition and composition and contribute to the accidental complexity associated with reactive system development.

\paragraph{Broader impact.}
Reactive systems have had a profound impact on society and will continue to impact society for the foreseeable future.
Some reactive systems like the Internet and smart phones have high visibility and continue to amaze while others, like the army of micro-controllers present in a modern automobile or a home appliance, are less conspicuous but nevertheless help us with our daily activities and contribute to our safety and comfort.
Some reactive systems, like pace makers, life support machines, and robotic surgical instruments, even have a direct impact on our health and well being.
The goal of this research is to ensure the quality and reliability of reactive systems in the face of predicted increases in size, diversity, and complexity.

\paragraph{Contributions.}
In this work, we propose three contributions to the state of the art in reactive systems development.
First, we propose a new model called reactive components for reactive systems based on direct support for reactive semantics and principled composition and decomposition.
A reactive component is a set of state variables and non-deterministically scheduled atomic transitions.
Transitions in different components can be linked via ports which also allow data to be exchanged among components.
Implicit atomicity and non-deterministic sequencing allow reactive systems to be designed and implemented through principled decomposition and composition.

Second, we propose to implement the model of reactive components to determine whether the assumptions and semantics of the model can be realized using existing techniques and architectures.
For tractability, we limit the implementation to systems with a fixed configuration of reactive components.
The major challenge when implementing the model is to enforce the semantics that require all state transitions to be deterministic.
Our approach is based on the encoding of transitions using a pure functional or applicative language which in turn requires an approach to data structures (persistent vs. ephemeral).
The platform will consist of a compiler for a high-level language shaped by the semantics of reactive components and a virtual machine.

Third, we propose to evaluate the model of reactive components and its implementation.
In the first part of the evaluation, we will apply the model to the design and development of an embedded web server.
The goal of the evaluation is to gauge the fitness of the model and usefulness of principled composition and decomposition by applying the model to a representative real-world reactive system.
Developing an embedded web server allows the model to be applied to a number of problems in reactive systems such as hardware, network protocols, and interactive applications.
The second part of the evaluation is a quantitative measurement of the implementation (compiler and virtual machine) to ensure that the employed algorithms and techniques result in a practical engineering tool.

\paragraph{Timeline.}
We propose the following timeline for the different contributions:  4~months for the model (section~\ref{model}), 12~months for the implementation (section~\ref{implementation}), and 5~months for the evaluation (section~\ref{evaluation}).

%%\paragraph{Future work.}
\paragraph{Beyond this dissertation.}
While we argue that the static system assumption is valid, it is nevertheless restrictive and prevents designs that may be more naturally formulated using a unbound number of components.
Thus, we believe the first step to applying reactive components to a broader range of systems is an extension to dynamic systems.
Another barrier to the wide-spread adoption of reactive systems is its reliance on a virtual machine.
As indicated by the Java programming language, a virtual machine is appropriate for application level programming while lower-level programming is often performed in a compiled language such as C.
Thus, the compiler proposed in section~\ref{implementation} may serve as the beginning for a compiler capable of emitting machine code.
The main direction for future work is to continue to gain experience with reactive components by implementing systems.
The class of reactive systems is enormous and spans everything from the programs that control 8-bit micro-controllers to the distributed applications running Google's cluster.
We believe the evaluation of section~\ref{evaluation} is an appropriate first case study but necessarily lacks the depth and breadth needed to fully evaluate reactive components.

\paragraph{Outcomes.}
The proposed research will be successful if it generates new knowledge regarding the design and implementation of reactive systems.
The conclusion of this research will be a explanation as to whether or not reactive components are a viable approach to reactive systems and if so, what additional research is warranted.
If the research concludes that reactive components are not a viable approach, it will reveal the characteristics of reactive components and more generally models based on the non-deterministic sequencing and implicit atomicity that make them an inappropriate or impractical foundation for reactive systems.
If reactive components are indeed a viable approach to reactive systems, then the proposed work could represent the beginning of a significant shift in the theory and practice concerning reactive systems.
That is, reactive components have the potential to replace threads and events which are the dominant approaches to reactive system development today.
