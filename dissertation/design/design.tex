\section{Design of the Runtime System \label{design}}

\paragraph{System images.}
A \emph{system image} is a serialized but executable description of a system of reactive components.
A system image is a virtual machine image.
I envision a number of tools that operate on system images:
\begin{itemize}
\item \emph{Compilers} that translate system descriptions in high-level languages into system images.
\item \emph{Interpretters} execute system images.
\item \emph{Verifiers} determine if all action trees are deterministic.
\item \emph{Parallelizers} determine which action trees can be executed in parallel.
\item Various other tools for optimizing, profiling, debugging, and modifying system images.
\end{itemize}

A system image contains the following items:
\begin{itemize}
\item type descriptors for all types in the system
\item the list of components and their state
\item the list of bindings
\item code for all preconditions, effects, functions, methods, etc.
\item scheduler metadata
\end{itemize}

\paragraph{Interpretter.}
The interpretter is the central component of the runtime system.
I assume that it will also perform the work of a verifier and parallelizer.
The basic function of the interpretter is as follows:
\begin{enumerate}
\item Read the type information, the list of components, the list of bindings, and the code, and verify that all actions are deterministic.
\item If using a parallel scheduler, determine which actions can be executed in parallel.
\item Recreate the state of each component in memory.
\item Start scheduling and executing actions.
\item Optionally record the final state of each component and exit.
\end{enumerate}

\paragraph{Proposed classes, interfaces, and algorithms.}
\begin{description}
\item[SystemImageReaderWriter] Reads/writes the various sections of a system image.
\item[TypeDescriptor] Class that describes user define types.
\item[Component] A tuple of type and address.
\item[Binding] A tuple of component-port and component-reaction.
\item[Procedure] Class representing executable code.
\item[Verify] Takes a component, action, and the list of bindings and determines if the action is deterministic.
\item[Parallelize] Takes a list of components and bindings and determines if the actions are independent.
\item[MemoryMap] Address/value pairs representing the contents of memory.
\item[MemoryManager] Class that can recreate an object from its MemoryMap.  Also contains the garbage collector.
\item[Scheduler] Class that selects actions for execution.
\end{description}

\paragraph{Sketch of Verify and Parallelize.}
Verify and Parallize are actually different ways of calling and interpretting the result of a more general algorithm I will call \emph{Deterministic}.
The sketch of Deterministic is based on the concept of \emph{memory regions} which are a technique used in pointer analysis.
A memory region is the set of addresses that are semantically reachable from a typed pointer.

\begin{itemize}
\item The address associated with a non-pointer, scalar value is a memory region.

\item An address containing a value that is a pointer
\end{itemize}

Is this a data-flow problem?

A memory region is a abstract set of addresses such that all pointers have a value pointing inside the region.
A memory region may be decomposed into smaller memory regions as follows:

\begin{verbatim}
S is set of pairs where each pair is <component, Set of actions/reactions>
iterator over the action tree to generate S
for each element s in S
  if size (s.second) > 1

  endif
endfor
\end{verbatim}

\paragraph{Sketch of Parallelize.}
