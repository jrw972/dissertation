\documentclass[letterpaper]{book}

%%\usepackage{cite}
%%\usepackage{pstricks}
\usepackage{url}
\usepackage{tikz}
\usepackage{float}
\usepackage{graphicx}
\usepackage{cprotect}
\usepackage{amsmath}
\usepackage{longtable}
\usepackage{fancyvrb}

\usetikzlibrary{shapes, calc, patterns, arrows, decorations.pathreplacing, decorations.markings, positioning, automata}

%%\newtheorem{theorem}{Theorem}

\title{A Model and Platform for Designing and Implementing Reactive Systems}
\author{Justin R. Wilson}
\date{}

\begin{document}

\VerbatimFootnotes

\maketitle

\input introduction.tex

\input background.tex

\input model.tex

\input language.tex

\input runtime.tex

\input conclusion.tex

\bibliographystyle{plain}
\bibliography{paper}{}

\appendix

\chapter{Partition Tables}

\begin{longtable}{ccccccccr}
Symbol & System & Server & Counter & Client & Response & Tick & Request & Count \\
\hline
\endhead
A & 0 & 0 & 0 & 0 & 0 & 0 & 0 & 17 \\
B & 0 & 0 & 0 & 0 & 0 & 0 & 1 & 16 \\
C & 0 & 0 & 0 & 0 & 0 & 1 & 0 & 20 \\
D & 0 & 0 & 0 & 0 & 0 & 1 & 1 & 12 \\
E & 0 & 0 & 0 & 0 & 1 & 0 & 0 & 11 \\
F & 0 & 0 & 0 & 0 & 1 & 0 & 1 & 17 \\
G & 0 & 0 & 0 & 0 & 1 & 1 & 0 & 11 \\
H & 0 & 0 & 0 & 0 & 1 & 1 & 1 & 12 \\
I & 0 & 0 & 0 & 1 & 0 & 0 & 0 & 13 \\
J & 0 & 0 & 0 & 1 & 0 & 0 & 1 & 6 \\
K & 0 & 0 & 0 & 1 & 0 & 1 & 0 & 17 \\
L & 0 & 0 & 0 & 1 & 0 & 1 & 1 & 20 \\
M & 0 & 0 & 0 & 1 & 1 & 0 & 0 & 16 \\
N & 0 & 0 & 0 & 1 & 1 & 0 & 1 & 25 \\
O & 0 & 0 & 0 & 1 & 1 & 1 & 0 & 14 \\
P & 0 & 0 & 0 & 1 & 1 & 1 & 1 & 12 \\
Q & 0 & 0 & 1 & 0 & 0 & 0 & 0 & 14 \\
R & 0 & 0 & 1 & 0 & 0 & 0 & 1 & 11 \\
S & 0 & 0 & 1 & 0 & 0 & 1 & 0 & 9 \\
T & 0 & 0 & 1 & 0 & 0 & 1 & 1 & 15 \\
U & 0 & 0 & 1 & 0 & 1 & 0 & 0 & 20 \\
V & 0 & 0 & 1 & 0 & 1 & 0 & 1 & 12 \\
W & 0 & 0 & 1 & 0 & 1 & 1 & 0 & 18 \\
X & 0 & 0 & 1 & 0 & 1 & 1 & 1 & 16 \\
Y & 0 & 0 & 1 & 1 & 0 & 0 & 0 & 16 \\
Z & 0 & 0 & 1 & 1 & 0 & 0 & 1 & 16 \\
a & 0 & 0 & 1 & 1 & 0 & 1 & 0 & 18 \\
b & 0 & 0 & 1 & 1 & 0 & 1 & 1 & 21 \\
c & 0 & 0 & 1 & 1 & 1 & 0 & 0 & 16 \\
d & 0 & 0 & 1 & 1 & 1 & 0 & 1 & 9 \\
e & 0 & 0 & 1 & 1 & 1 & 1 & 0 & 14 \\
f & 0 & 0 & 1 & 1 & 1 & 1 & 1 & 11 \\
g & 0 & 1 & 0 & 0 & 0 & 0 & 0 & 16 \\
h & 0 & 1 & 0 & 0 & 0 & 0 & 1 & 17 \\
i & 0 & 1 & 0 & 0 & 0 & 1 & 0 & 12 \\
j & 0 & 1 & 0 & 0 & 0 & 1 & 1 & 20 \\
k & 0 & 1 & 0 & 0 & 1 & 0 & 0 & 16 \\
l & 0 & 1 & 0 & 0 & 1 & 0 & 1 & 16 \\
m & 0 & 1 & 0 & 0 & 1 & 1 & 0 & 13 \\
n & 0 & 1 & 0 & 0 & 1 & 1 & 1 & 17 \\
o & 0 & 1 & 0 & 1 & 0 & 0 & 0 & 25 \\
p & 0 & 1 & 0 & 1 & 0 & 0 & 1 & 22 \\
q & 0 & 1 & 0 & 1 & 0 & 1 & 0 & 13 \\
r & 0 & 1 & 0 & 1 & 0 & 1 & 1 & 14 \\
s & 0 & 1 & 0 & 1 & 1 & 0 & 0 & 10 \\
t & 0 & 1 & 0 & 1 & 1 & 0 & 1 & 11 \\
u & 0 & 1 & 0 & 1 & 1 & 1 & 0 & 27 \\
v & 0 & 1 & 0 & 1 & 1 & 1 & 1 & 16 \\
w & 0 & 1 & 1 & 0 & 0 & 0 & 0 & 12 \\
x & 0 & 1 & 1 & 0 & 0 & 0 & 1 & 18 \\
y & 0 & 1 & 1 & 0 & 0 & 1 & 0 & 17 \\
z & 0 & 1 & 1 & 0 & 0 & 1 & 1 & 12 \\
0 & 0 & 1 & 1 & 0 & 1 & 0 & 0 & 17 \\
1 & 0 & 1 & 1 & 0 & 1 & 0 & 1 & 7 \\
2 & 0 & 1 & 1 & 0 & 1 & 1 & 0 & 11 \\
3 & 0 & 1 & 1 & 0 & 1 & 1 & 1 & 17 \\
4 & 0 & 1 & 1 & 1 & 0 & 0 & 0 & 19 \\
5 & 0 & 1 & 1 & 1 & 0 & 0 & 1 & 19 \\
6 & 0 & 1 & 1 & 1 & 0 & 1 & 0 & 16 \\
7 & 0 & 1 & 1 & 1 & 0 & 1 & 1 & 17 \\
8 & 0 & 1 & 1 & 1 & 1 & 0 & 0 & 21 \\
9 & 0 & 1 & 1 & 1 & 1 & 0 & 1 & 22 \\
@ & 0 & 1 & 1 & 1 & 1 & 1 & 0 & 22 \\
\$ & 0 & 1 & 1 & 1 & 1 & 1 & 1 & 13 \\

\caption{Partitions for the AsyncClock system.  The Symbol column contains the symbol used to represent this partition on plots.  The System, Server, Counter, and Client columns indicate the thread used for the garbage collection action for the respective component.  The Response, Tick, and Request columns indicate the thread used for the respective transaction.  The Count column indicates the number of samples for this partition.}
\label{async_partitions}
\end{longtable}

\begin{longtable}{ccccccccr}
Symbol & System & Counter & Request & Tick & Count \\
\hline
\endhead
A & 0 & 0 & 0 & 0 & 117 \\
B & 0 & 0 & 0 & 1 & 137 \\
C & 0 & 0 & 1 & 0 & 119 \\
D & 0 & 0 & 1 & 1 & 114 \\
E & 0 & 1 & 0 & 0 & 117 \\
F & 0 & 1 & 0 & 1 & 140 \\
G & 0 & 1 & 1 & 0 & 135 \\
H & 0 & 1 & 1 & 1 & 121 \\

\caption{Partitions for the SyncClock system.  The Symbol column contains the symbol used to represent this partition on plots.  The System and Counter columns indicate the thread used for the garbage collection action for the respective component.  The Request and Tick columns indicate the thread used for the respective transaction.  The Count column indicates the number of samples for this partition.}
\label{sync_partitions}
\end{longtable}

\end{document}
